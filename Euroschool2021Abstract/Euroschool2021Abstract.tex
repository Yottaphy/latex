% Abstract template for Physics Days 2020
\documentclass[12pt, a4paper]{article}
\usepackage{fp2021}

\begin{document}

%%%%%%%%%%%%%%%%%%%%%%%%%%%%%%%%%%%%%%%%%%%%%%%%%%%%%%%%%%%%%%%%%%%%%%%%%%%%%%%%
% Header 
%%%%%%%%%%%%%%%%%%%%%%%%%%%%%%%%%%%%%%%%%%%%%%%%%%%%%%%%%%%%%%%%%%%%%%%%%%%%%%%%

% Title in capital letters
\title{STATUS OF THE MARA-LEB FACILITY}

% List of authors with the presenting authors name underlined
{\underline{J.~Romero}\textsuperscript{1,2,*}, W.~Gins\textsuperscript{1}, I.~D.~Moore\textsuperscript{1}, P.~Papadakis\textsuperscript{3}, J.~Sarén\textsuperscript{1}, J.~Uusitalo\textsuperscript{1} and A.~Zadvornaya\textsuperscript{1}}.

% Affiliations
{\textsuperscript{1}\em Department of Physics, University of Jyväskylä, Finland}

{\textsuperscript{2}\em Department of Physics, University of Liverpool, United Kingdom}

{\textsuperscript{3}\em Nuclear Physics Group, STFC Daresbury Laboratory, United Kingdom}

% Presenting authors email
*Email: \href{mailto:joromero@jyu.fi}{{joromero@jyu.fi}}

% Extra space after header
\vspace{2\baselineskip}

%%%%%%%%%%%%%%%%%%%%%%%%%%%%%%%%%%%%%%%%%%%%%%%%%%%%%%%%%%%%%%%%%%%%%%%%%%%%%%%%
% Main text
%%%%%%%%%%%%%%%%%%%%%%%%%%%%%%%%%%%%%%%%%%%%%%%%%%%%%%%%%%%%%%%%%%%%%%%%%%%%%%%%

MARA-LEB (Mass Analysing Recoil Apparatus - Low Energy Branch) is a facility currently in the final stages of design as an extension to the MARA separator at the Accelerator Laboratory of the University of Jyväskylä \cite{MARA}. The aim of this new facility is to study exotic ions close to the proton drip line and the region near the N=Z=50 doubly-magic shell closure. This region, which includes the heaviest self-conjugate ions, such as \ce{^{80}Zr}, \ce{^{94}Ag} and  \ce{^{100}Sn}, is optimal for the testing of nuclear models and their predictions \cite{theo}. In addition to this, because many nuclei play a significant role in the astrophysical rapid proton (rp) capture process, this region of the nuclear chart is of importance in astrophysical nucleosynthesis models \cite{astro}. 

The combination with the existing MARA facility is ideal due to the high mass selectivity of the separator \cite{josh}. Mass-selected recoils from MARA enter the first part of the facility, the gas cell, through a thin window that separates a buffer gas region from MARA's high-vacuum environment. Incoming recoils will be stopped and neutralised by a buffer gas within the cell. The neutralisation of the recoils allows for subsequent in-gas-cell or in-gas-jet laser ionisation and spectroscopy via a state-of-the-art Ti:Sapphire laser system \cite{conf}.  

The ions emerging from the gas cell will be transported and focused towards the experimental stations by the MARA-LEB ion transport system \cite{ion}. This will be achieved by a 90º-bent radio-frequency quadrupole (RFQ), followed by a straight, segmented RFQ, which will transport the ions to acceleration electrodes. These will accelerate the ions to 30\,keV and feed them into a magnetic dipole which will provide further mass separation for contaminant suppression. 

An experiment was carried out using the MARA separator to investigate the production of isotopes in MARA-LEB's region of interest close to \ce{^{96}Ag}. The charge state distribution of \ce{^{96}Pd} recoils at the focal plane of MARA has helped to finalise design aspects concerning the size of the gas cell window. The assembling of the beamline is also underway, which will allow for the experimental verification of previous simulations. In this contribution, a general overview of the MARA-LEB facility and its current state, along with preliminary results of the aforementioned experiment, will be presented.



%\begin{figure}[h]
%\centering
%\includegraphics[width=10cm]{Fig1.pdf}
%\caption{Finding some correlations.}
%\end{figure}


%%%%%%%%%%%%%%%%%%%%%%%%%%%%%%%%%%%%%%%%%%%%%%%%%%%%%%%%%%%%%%%%%%%%%%%%%%%%%%%%
% Bibliography
%%%%%%%%%%%%%%%%%%%%%%%%%%%%%%%%%%%%%%%%%%%%%%%%%%%%%%%%%%%%%%%%%%%%%%%%%%%%%%%%

% Extra space after main text
\vspace{2\baselineskip}

\begin{thebibliography}{9}
\bibitem{MARA} J.~Sarén, et al., {\it Nucl.~Inst.~Methods~Phys.~Res.~B} {\bf 266}, 4196 (2008). 
\bibitem{theo} T.~Faestermann, M.~Gorska, and H.~Grawe, {\it Prog.~Part.~Nucl.~Phys.} \textbf{69}, 85 (2013).
\bibitem{astro} R.~K.~Wallace, and S.~E.~Woosley, {\it
Astrophys. J. Suppl. Ser.}, \textbf{45}, 398 (1981).
\bibitem{josh} J.~Uusitalo, J.~Sarén, J.~Partanen, and J.~Hilton, {\it Acta~Phys.~Pol.~B} {\bf 50}, 319 (2019).
\bibitem{conf} P.~Papadakis, et al., {\it AIP~Conf.~Proc.} \textbf{2011}, 070013 (2018).
\bibitem{ion} P.~Papadakis, et al., {\it Nucl.~Inst.~Methods~Phys.~Res.~B} \textbf{463}, 286 (2020).
\end{thebibliography}

\end{document}
