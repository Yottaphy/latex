\section{Recoil separators, reactions, energy losses}
\newcommand{\la}{\ce{^{120}La}}


\textbf{a)} To select the appropriate beam and target combination to use to produce \ce{^{120}La} in MARA, firstly, we need to determine what the possible beam compositions are. For that, we can visit the ECR Beam Database \cite{beams} of the University of Jyväskylä's Ion Source Group.

To produce \la\ through fusion-evaporation, a suitable compound nucleus has to be produced, which will be more massive than \la. Candidates include \ce{^{122}Ce} (decaying to \la\ through the pn channel), \ce{^{123}Ce} (p2n channel), \ce{^{124}Pr} (2p2n channel), and other isotopes of lanthanum with a higher A, decaying through the xn channel, where x is A-120.

To produce a compound in this region, a target of mass around A=65 is needed, with a beam of mass around A=60. By analysing the possible beams in \cite{beams} and the corresponding targets, with the condition that the target material should be a stable nucleus, the following reaction is chosen: a \ce{^{58}_{28}Ni_{30}} beam impinging on a \ce{^{66}_{30}Zn_{36}} target. This produces a compound nucleus of \ce{^{124}_{58}Ce^*_{76}}, which then can decay through the p3n channel to produce \ce{^{120}La}. 
\begin{equation}
    \label{reaction}
    \ce{^{66}Zn} + \ce{^{58}Ni} \rightarrow \ce{^{124}Ce^*} \rightarrow \ce{^{120}La} + p + 3n.
\end{equation}

The reaction of \ce{^{58}Ni} onto \ce{^{63}Cu} to produce \ce{^{121}La^*} and evaporate one proton was also considered, but the fusion cross section was too low for the energies in which only one proton would be evaporated, making it almost impossible to get to \la\ with this particular reaction.

\textbf{b)} To calculate the required bombarding energy, the separation energies of the intermediate nuclei have to be taken into account, to calculate the excitation energy, $E^*$, of the compound nucleus. These energies are shown in \autoref{tab:separationE}:
\begin{table}[H]
    \caption{Relevant separation energies for the compound nucleus and intermediate nuclei \cite{ame}.}
    \label{tab:separationE}
    \begin{tabular}{@{}ccc@{}}
        \hline
        Nucleus & $S_n$ [MeV] & $S_p$ [MeV]\\ \hline
        \ce{^{124}Ce} &  -& 0.269\\
        \ce{^{123}La} & 12.179 & -\\
        \ce{^{122}La} & 10.424 & -\\
        \ce{^{121}La} & 12.692 & -\\ \hline 
        \textbf{Total} & 35.295 & 0.269 \\ \hline
    \end{tabular}
\end{table} 

The energies in \autoref{tab:separationE} amount to a total separation energy of 35.564\unit{MeV}. In addtion, an average of 3\unit{MeV} of kinetic energy per emitted particle has to be taken into account. Since 4 nucleons will be emitted, 12\unit{MeV} have to be added, obtaining an excitation energy of $E^* = 47.6\unit{MeV}$. 

It is now necessary to calculate the reaction Q-value, from their mass excess. The fusion Q-value will be:
\begin{equation}
    \label{qvalue}
    Q_{fus} = -\Delta M(\text{compound}) + \Delta M(\text{target})+ \Delta M(\text{beam}),
\end{equation}
where $\Delta M$ represents the mass excess. \autoref{tab:masses} shows the mass excess for each of these nuclei according to \cite{ame}.

\begin{table}[H]
    \caption{Mass excess for the nuclei involved in the reaction in \autoref{reaction}~\cite{ame}.}
    \label{tab:masses}
    \begin{tabular}{@{}cc@{}}
        \hline
        Nucleus & $\Delta M$ [MeV] \\ \hline
        \ce{^{124}Ce} &  -64.916\\
        \ce{^{66}Zn}  &  -68.899\\
        \ce{^{58}Ni}  &  -60.229\\ \hline 
    \end{tabular}
\end{table}
The reaction Q-value, calculated using \autoref{qvalue}, is $Q_{fus} = -64.212\unit{MeV}$. Since the centre-of-mass energy is given by $E_{CM} = E^* - Q$, for this reaction $E_{CM} = 111.8\unit{MeV}$. By a simple conversion:
\begin{equation}
    {E_{lab} \approx \frac{A_A + A_B}{A_A}E_{CM}} \Rightarrow {E_{lab} =   \frac{66+58}{66}\times 111.8 = 210.1\unit{MeV}}.
\end{equation}

Calculating the Coulomb barrier energy as:
\begin{equation}
    {V_C = \frac{Z_1 Z_2}{A_1^{1/3}+ A_2^{1/3}} = \frac{28\cdot30}{58^{1/3}+66^{1/3}}=106.2\unit{MeV}},
\end{equation}  
we can see that the reaction is well above the Coulomb barrier, and therefore can occur. 

The energy at the middle of the target (MOT) will be 210\unit{MeV}. Since the target is assumed to be 0.7\unit{mg/cm$^2$}, this means that the bombarding energy has to jjjj such that the remaining energy after 0.35\unit{mg/cm$^2$} of target is 210\unit{MeV}. From SRIM \cite{srim}, we can obtain the stopping power of \ce{^{58}Ni} in \ce{^{66}Zn}, which is 20.71\unit{MeV/(mg/cm$^2$)} at 225\unit{MeV} beam energy. With this in mind, the energy loss in half of the target's thickness will be: $$20.71\unit{MeV/(mg/cm$^2$)}\times0.35\unit{mg/cm$^2$} = 7.25\unit{MeV},$$ making it so that the optimal bombarding energy is \textbf{217.25\unit{MeV}}. This was also verified with the LISE++ code \cite{lise}.
\newpage
\textbf{c)} To calculate the magnetic field and electric potential needed for \la's transport through MARA, it is necessary to obtain its kinetic energy after the target. It is possible to calculate its energy at the middle of the target, $T_{MOT}$, thanks to the conservation of angular momentum, as:
\begin{equation}
    \label{MOT}
    T_{MOT} = E_{lab}\cdot\frac{A_{beam}}{A_{comp}}\cdot\frac{A_{recoil}}{A_{comp}} = 210\unit{MeV}\frac{58\cdot 120}{124^2} = 95\unit{MeV}.
\end{equation}
After this, the stopping power of \la\ in \ce{^{66}Zn} is needed, as this is the MOT energy. The recoils will still need to go through 0.35\unit{mg/cm$^2$} of target. Additionally, MARA has a thin, 20\unit{\textmu g/cm$^{2}$} carbon foil after the target position, which the recoil will also need to traverse. 

The stopping power of \la\ in \ce{^{66}Zn} at 95\unit{MeV} is 28\unit{MeV/(mg/cm$^2$)}. Thus, after 0.35\unit{mg/cm$^2$}, the recoils would have deposited 10\unit{MeV} and be left with 85\unit{MeV}.

The stopping power of \la\ in \ce{^{nat}C} at 85\unit{MeV} is 57\unit{MeV/(mg/cm$^2$)}. The thickness of the C foil is 0.02\unit{mg/cm$^2$}, so the recoil will deposit 1.1\unit{MeV} and therefore leave with a final kinetic energy of $T = 83.9\unit{MeV}$.

It is now possible to calculate the recoil velocity from its kinetic energy, as: 
\begin{equation}
    \label{eq:velocity}
    T = \frac{1}{2}mv^2 \Longrightarrow v = \sqrt{\frac{2T}{m}}.
\end{equation}
In this case, with the recoil's mass being $m = 120\unit{u}$, it is possible to obtain the velocity in terms of the speed of light as:
\begin{equation}
    \label{eq:vel_number}
    v = \sqrt{\frac{2\cdot83.9\unit{MeV}}{120\unit{u}\cdot931.5\unit{MeV/(uc$^2$)}}} = 0.0387\unit{c}.   
\end{equation}
Which can be used to calculate the reduced speed with $v' = 0.012\unit c$ \cite{ND}:

\begin{equation}
    \label{eq:reduced}
    x = \frac{v}{v'+Z^{0.45}} = 0.523,
\end{equation}
which serves to estimate the average charge state in which the recoils will leave. This will be calculated as \cite{ND}:
\begin{equation}
    \label{eq:charge}
    \bar q = Z\left[1+ x^{-1/0.6}\right]^{-0.6} = 25.
\end{equation}

Also from \cite{ND}, we can find the distribution of charge states as:
\begin{equation}
    \label{eq:dist}
    d_{\bar q} = \frac{1}{2}\sqrt{\bar q \left[1-\left(\frac{\bar q}{Z}\right)^{1/0.6}\right]} = 2.16 \rightarrow 2,
\end{equation}
which means that, although charge state 25 is the average and more commonly produced, charge states 24 and 26 will also be produced.

Having this, the values for the deflector electric potential, $V$, and the magnetic dipole field, $B$, for MARA to focus this charge state to the centre of the focal plane will be:
\begin{align}
    V &= \frac{T}{q}\ln{\left(\frac{R_2}{R_1}\right)},    \label{eq:voltage}\\
    B &= \frac{\sqrt{2mT}}{q\rho_B},         \label{eq:magnetic}
\end{align}
where $T$ is the alpha energy in MeV, $m$ is the mass of the recoil in u (120 in this case), $q$ is the charge of the particle (25 in this case), $\rho_B = 1\unit m$ is the magnetic radius and $R_2$ and $R_1$ are the outer and inner electrode radii, respectively, where, since MARA has a symmetric setup, $R_{\substack{1\\2}}  = \rho_E \mp 0.07$, as the half-radius of the deflector is subtracted or added to the electric radius, $\rho_E = 4\unit{m}$ \cite{saren}. 

Solving these equations with the derived and given values, the resulting MARA optical tuning parameters to have the possible charge states centred at the focal plane are shown in \autoref{tab:maraparams}. 

\begin{table}[H]
    \caption{MARA Optical Parameters for \la\ at 83.9\unit{MeV} and relevant charge states.}
    \label{tab:maraparams}
\begin{tabular}{ccc}
    \hline
    Charge State & V[V] & B[T]\\ \hline
    24 & 122\,367 & 0.602\\
    25 & 117\,472 & 0.578\\
    26 & 112\,954 & 0.556\\ \hline
\end{tabular}
\end{table}

\textbf{d) } By calculating the values of $m/q$ for the three relevant charge states shown in the previous section (q = {24,25,26}), and using $m=120$, it can be seen in \autoref{tab:chstates}, a one unit charge state difference with respect to $q=25$ results in a $\sim 4\%$ change in $m/q$.

\begin{table}[H]
    \caption{\la\ $m/q$ values for relevant charge states and their percentage difference with respect to $\bar q = 25$.}
    \label{tab:chstates}
\begin{tabular}{ccr}
    \hline
    Charge State & $m/q$ & \multirow{1}{*}{$\Delta(m/q)$ [\%]}\\ \hline
    24 & 5.000 & +4.17\\
    25 & 4.800 & 0.00\\
    26 & 4.615 & -3.85\\ \hline
\end{tabular}
\end{table}

Due to this, and considering the $8.00\unit{mm/(\%\ m/q)}$, charge states will be separated by 32\unit{mm} at the MWPC. Since the MWPC is 128\unit{mm} wide, with charge state 25 centred at its middle, there are 64\unit{mm} available at each side of the central charge state. Therefore, 2 additional charge states can fit the MWPC on each side of the central one, although the ones on the extremes may lose some counts if not very well focused. Therefore, we can say that we would expect \textbf{5 charge states within the MWPC} under ideal conditions.

\textbf{e) } Firstly, to calculate the angular cone, it is necessary to identify the three angular components of the recoils:
\begin{enumerate}
    \item Multiple scattering in the target foil, $\alpha_{1/2}^{scatt}$,
    \item Angular component due to particle emission, $\alpha_{1/2}^{part}$,
    \item Beam emittance from the cyclotron, $\alpha_{1/2}^{beam}$.
\end{enumerate}

The beam emittance from the cyclotron is fixed at $\alpha_{1/2}^{beam} = 5\unit{mrad}$, so only the other two components need to be calculated. 

For the first one, it is necessary to calculate the reduced angles: 
\begin{align*}
    \label{eq:reduced}
    \tilde \alpha &= \frac{aE_p}{Z_pZ_t(Z_p^{2/3}+Z_t^{2/3})^{1/2}}\alpha, \\
    \tilde{\alpha}_{1/2} &\approx 0.25\tau,\\
    \tau &= \frac{41.5\rho}{A_t(Z_p^{2/3}+Z_t^{2/3})},
\end{align*} where $a=0.885a_0\left(Z_p^{2/3}+Z_t^{2/3}\right)^{-1/2}$, $a_0$ is Bohr's radius, $\rho$ is the thickness of the target and $t$ and $p$ in subscripts represent target and projectile, respectively.

Combining all of these, we can arrive at:
\begin{equation}
    \label{eq:scatter}
    \alpha_{1/2}^{scatt} = \frac{11.723\cdot Z_pZ_t\rho}{a_0E_pA_t}.
\end{equation}  

The projectile energy is $T_{MOT} = 95\unit{MeV}$, as calculated in \autoref{MOT}. In total, this amounts to $\mathbf{\alpha_{1/2}^{scatt} = 13.8\unit{\textbf{mrad}}}$.

The angular component due to particle emission, $\alpha_{1/2}^{part}$, is defined as:
\begin{equation}
    \label{eq:particle}
    \alpha_{1/2}^{part} = \frac{2.36}{2} \sigma = \frac{2.36}{2} \sqrt{\Omega_n},
\end{equation} where 
\begin{equation}
    \label{eq:omega}
    \Omega_n = \sum_i \frac{p_i^2}{3p_{comp}^2}.
\end{equation}
$p_i$ and $p_{comp}$ are the momenta of the emitted neutrons and the compound nucleus, respectively. Since $p^2 = \sqrt{2Em}$, and all 4 emitted particles are emitted with 3\unit{MeV} and have the same mass ($m_p \approx m_n$), \autoref{eq:omega} can be rewritten as:
\begin{equation}
    \label{eq:omega2}
    \Omega_n = \frac{4\times 3\unit{MeV}\times 1\unit{u}}{3\times 98.2\unit{MeV}\times 124\unit{u}},
\end{equation} due to $E_{comp} = \frac{A_{beam}}{A_{comp}}E_{lab} = \frac{58}{124}\times 210\unit{MeV} = 98.2\unit{MeV}$. In total, substituting \autoref{eq:omega2} into \ref{eq:particle}, we obtain the final angular component due to particle emission, $\mathbf{\alpha_{1/2}^{part} = 21.48\unit{\textbf{mrad}}}$.

Combining all of these, the total angular cone turns out to be:
\begin{equation}
    \label{eq:cone}
    \boxed{\alpha_{1/2} = \sqrt{13.8^2 + 21.4^2 + 5^2} =  26\unit{mrad}.}
\end{equation}

The angular acceptance of MARA is $\pm 50\unit{mrad}$ when considering both x and y directions, so the entire cone will be accepted.

Finally, to calculate the energy straggling, it is necessary to calculate the individual components of energy straggling. These are:

\textbf{i.} Beam energy distribution. This is calculated as $\Delta E_{beam} = 0.01 E_{beam}$, in this case, $\Delta E_{beam} = 2.1\unit{MeV}$.

\textbf{ii.} Broadening due to thickness, calculated as:
\begin{equation}
    \label{eq:thickness}
    \Delta E_{thick} = \abs{\delta E_{rec} + \delta E_{beam}\frac{A_{beam}}{A_{rec}}}.
\end{equation}

In this case,  $\delta E_{beam} = 14.5\unit{MeV}$ and $\delta E_{rec} = 10\unit{MeV}$, so $\Delta E_{thick} =17\unit{MeV}$.

\textbf{iii.} Broadening due to straggling in the target foil, calculated as:
\begin{equation}
    \label{eq:straggling}
    \Delta E_{strag} = 2.89\E{-2}\left[\delta E_{beam} Z_{beam}Z_{t}\left(Z_{beam}^{1/3}+Z_t^{1/3}\right)^{-2} + \delta E_{rec} Z_{rec}Z_{t}\left(Z_{rec}^{1/3}+Z_t^{1/3}\right)^{-2}\right]^{1/2}.
\end{equation}

In total, when substituting in, the result is $ \Delta E_{strag} =1.25\unit{MeV}$.

\textbf{iv.} Broadening due to particle emission, calculated as:
\begin{equation}
    \label{eq:emission}
    \Delta E_{part} = \frac{2\sqrt{8\ln 2}}{\sqrt{3}}\sqrt{\bar E_{rec}\frac{\bar E_n}{A_{rec}}},
\end{equation} where $\bar E_n$ and $\bar E_{rec}$ are the average energies for the nucleons emitted and the recoil, respectively. $\bar E_n = 3\unit{MeV}$ and $\bar E_{rec}$ can be calculated as:
\begin{equation}
    \label{eq:recavg}
    \bar E_{rec} = (A_{beam}\frac{A_{rec}}{A_{comp}^2})(E_{beam}-0.5\delta E_{beam}) - 0.5\delta E_{rec}.
\end{equation}

In this case, $ \bar E_{rec} = 99.4\unit{MeV}$ and thus, $\Delta E_{part} = 7.4\unit{MeV}$.

In total now, we can calculate the total energy straggling:
\begin{equation}
    \label{eq:energy}
    {2\delta_T = \sqrt{\Delta E_{beam}^2 + \Delta E_{thick}^2 + \Delta E_{strag}^2 + \Delta E_{part}^2} = 18.7\unit{MeV}} \Rightarrow \boxed{\delta_T = 9.4\unit MeV.}
\end{equation}

Since the emerging kinetic energy is 83.9\unit{MeV}, this $\delta_T$ is 11\%. Since MARA's energy acceptance is 20\%, this recoil will be entirely accepted.