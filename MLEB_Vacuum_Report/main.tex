\documentclass[12pt, a4paper, spanish, british, twoside]{article}
\usepackage[
  margin=3.2cm,
  includefoot,
  footskip=30pt
]{geometry}
\usepackage{authblk}
%Fonts
\usepackage[T1]{fontenc}
\usepackage[utf8]{inputenc}
\usepackage[11pt]{moresize}
\usepackage{anyfontsize}
\usepackage{textgreek}
%Figures
\usepackage{graphicx}
\usepackage{wrapfig}
\usepackage{floatrow}
\usepackage{subfig}
\captionsetup[figure]{font=small,labelfont=small}\usepackage{transparent}
\usepackage{eso-pic}
\usepackage{chngcntr}
\graphicspath{{./images/}}
%Tables
\usepackage{tabu}
\usepackage{longtable}
\usepackage{multirow}
\usepackage{booktabs}
\usepackage{float}
\floatstyle{plaintop}
\restylefloat{table}
\renewcommand{\tablename}{Table}
\usepackage{color, colortbl}
%Maths
\usepackage{amsmath}
\usepackage{amssymb}
\usepackage{dsfont}
\usepackage{braket}
\usepackage{mathtools}
\usepackage{mhchem}
\usepackage{slashed}
\usepackage[makeroom]{cancel}
\usepackage{mathrsfs}
\usepackage{siunitx}

%unit marker
\newcommand{\unit}[1]{\,\text{#1}}

%scientific notation shorthand
\newcommand{\E}[1]{\times 10^{#1}}

%Absolute value
\DeclarePairedDelimiter\abs{\lvert}{\rvert}

%Norm
\DeclarePairedDelimiter\norm{\lVert}{\rVert}
\makeatletter
\let\oldabs\abs
\def\abs{\@ifstar{\oldabs}{\oldabs*}}
%
\let\oldnorm\norm
\def\norm{\@ifstar{\oldnorm}{\oldnorm*}}
\makeatother


%Trace
\DeclarePairedDelimiter\tr{\text{Tr}\lbrace}{\rbrace}

%Commutator
\DeclarePairedDelimiterX\com[2]{[}{]}{#1 , #2}
%Anti-commutator
\DeclarePairedDelimiterX\acom[2]{\lbrace}{\rbrace}{#1 , #2}

%Dagger
\newcommand{\+}{\dagger}
%identity
\newcommand{\id}{\mathbbm{1}}
%partial
\newcommand{\del}{\partial}
%mute proof title
\newcommand{\mute}{\unskip\nopunct}
%differential d
\AtBeginDocument{\renewcommand{\d}{\text{d}}}

%Page Formatting
\usepackage[en-GB]{datetime2}
\DTMlangsetup[en-GB]{ord=omit}
\usepackage{fancyhdr}
\usepackage{afterpage}
\usepackage{titlesec}
\titlelabel{\thetitle.\quad}
\usepackage{blindtext}
\usepackage[bottom]{footmisc}
\renewcommand*\footnoterule{}
\renewcommand{\thefootnote}{\fnsymbol{footnote}}

%Chapter formatting
\usepackage{titlesec, blindtext, color}
\definecolor{gray75}{gray}{0.6}
\newcommand{\hsp}{\hspace{15pt}}

\titleformat{\chapter}[display]{\vspace*{-1cm}\centering\Huge\bfseries}{\vspace*{-2em}\fontsize{100}{110}\selectfont \textcolor{gray75}{\thechapter}}{-5pt}{\vspace*{-1em}\centering\huge\bfseries\textsc}[ \vspace*{-1em}\textcolor{gray75}{\titlerule}\vspace*{-1.5em}]


\titleformat{\section}
  {\Large\bfseries}{\thesection.}{0.5em}{\vspace*{0em}}

\titleformat{\subsubsection}
  {\bfseries}{\thesubsubsection}{0em}{\vspace*{0em}}

\renewcommand\sectionmark[1]{\markright{\thesection\ #1}}
%Hyperlinks for sections
\usepackage[hidelinks]{hyperref}
\hypersetup{
    allcolors=black
}

%Blank Page
\newcommand\blankpage{%
    \null
    \thispagestyle{empty}%
    \newpage}
    
%Double sided header and footer
\setlength{\headheight}{15pt}
\pagestyle{fancy}
\fancyhf{}
\fancyhead[LE,RO]{FYSS6320}
\fancyhead[RE,LO]{\textsc{Jorge Romero}}
\fancyfoot[RO,LE]{\thepage}
\renewcommand{\footrulewidth}{1pt}

% Redefine the plain page style
\fancypagestyle{plain}{
  \fancyhf{}%
  \fancyfoot[RO,LE]{\thepage}
  \renewcommand{\headrulewidth}{0pt}% Line at the header invisible
  \renewcommand{\footrulewidth}{1pt}}% Line at the footer visible

%Spacing for paragraphs and titles
\setlength{\parskip}{0.5em}
\titlespacing\section{0pt}{3pt}{0pt}\relax
\titlespacing\subsection{0pt}{12pt plus 4pt minus 2pt}{-\parskip}\relax
\titlespacing\subsubsection{0pt}{12pt plus 4pt minus 2pt}{-\parskip}\relax
%Code
\usepackage{color, soul}
\usepackage{xcolor}
\usepackage{listings}

\definecolor{mBlue}{rgb}{0,0,0.6}
\definecolor{mGray}{rgb}{0.2,0.2,0.2}
\definecolor{mRed}{rgb}{0.6,0,0}
\definecolor{backgroundColour}{rgb}{0.92,0.92,0.92}

\lstdefinestyle{CStyle}{
    backgroundcolor=\color{backgroundColour},   
    commentstyle=\color{mBlue},
    keywordstyle=\color{mRed},
    numberstyle=\tiny\color{mGray},
    stringstyle=\color{magenta},
    basicstyle=\footnotesize,
    breakatwhitespace=false,         
    breaklines=true,                 
    captionpos=b,                    
    keepspaces=true,                 
    numbers=left,                    
    numbersep=5pt,                  
    showspaces=false,                
    showstringspaces=false,
    showtabs=false,                  
    tabsize=2,
    language=C
}

%Autoref labels
\renewcommand{\figureautorefname}{Fig.}
\renewcommand{\tableautorefname}{Table}
\renewcommand{\tablename}{Table}
\renewcommand{\sectionautorefname}{Sec.}
\renewcommand{\subsectionautorefname}{Sec.}
\renewcommand{\appendixautorefname}{Appendix}
\renewcommand{\contentsname}{Contents}

\begin{document}

\begin{titlepage}
	
    \begin{center}
        
        \LARGE{\textbf{FYSS3550}}\\
        \large{\textbf{Techniques for Nuclear and Accelerator-based Physics Experiments}}\\
        \vspace{0.15\textheight}
        \LARGE{\textbf{Home Exam}}\\
        \vspace{0.05\textheight}
        \large{\textbf{19.04.2021-30.04.2021}}\\
    \end{center}
    \vspace{0.08\textheight}
    \begin{center}
       \large{\textbf{Jorge Romero}}\\\large{\textbf{joromero@jyu.fi}}
    \end{center}
    \vspace{0.05\textheight}
    \centering

        \large{\textbf{Department of Physics}}\\
    \centering
    \vfill
    \includegraphics[width=0.5\textwidth]{jyu-keskitetty-kaksikielinen.eps}

\end{titlepage}

\pagenumbering{arabic}
\section{Introduction}
\label{sec:intro}
The Low Energy Branch (LEB) is a new facility, designed as an extension for the MARA (Mass Analysing Recoil Apparatus) deflector in the Accelerator Laboratory at the University of Jyväskylä.

MARA-LEB will serve as a system for additional isotopic and mass selection after MARA. One of MARA-LEB's most important components is its buffer gas cell. Within it, a noble gas (He or Ar) will flow at $\sim$500-1000\,mbar, which will stop and neutralise recoils coming from MARA at high charge states. The recoils will then be laser ionised and transported to detectors via radio-frequency quadrupoles (RFQ) and acceleration and transport electrodes, as shown in \autoref{fig:LEB_dia}. This laser ionisation effectively selects only one particular element from all incoming recoils from MARA. A magnetic dipole will perform additional mass separation before the detector station for further purification of the beam.

\begin{figure}[H]
    \centering
    \includegraphics[width=\textwidth]{LEB.pdf}    
    \caption[The MARA-LEB facility]{Diagram of the MARA-LEB facility, highlighting the gas cell and showing in-gas-cell laser ionisation, one of two laser ionisation methods possible in MARA-LEB.}
    \label{fig:LEB_dia}
\end{figure}

The transport electrodes and mass separator have to be kept in vacuum, due to the use of strong electromagnetic fields that may discharge. However, the ions exiting the gas cell do so within a gas jet at relatively high pressures. Since this introduction of gas into the system is vital to the operation of MARA-LEB, but components down the beamline have to work in high vacuum, a system of {differential pumping} was designed for this facility. This ensures that high and low pressure areas can coexist in the same beamline. 

In this report, both the normal and differential pumping regions will be discussed, with special focus on the latter, due to its higher complexity. 
\newpage

\section{High Vacuum at MARA}
\label{sec:mara}
The first component of the LEB facility is the gas cell, which, as previously explained, will have a buffer gas flowing at pressures of 500-1000\,mbar. The gas cell will be placed at the MARA focal plane, to receive the mass-separated recoils coming from the deflector. MARA, however, is kept at a high vacuum of about $10^{-7}$\,mbar. Therefore, there must be a barrier separating these two regions, which should be strong enough to withstand pressure differences of up to  $10^{10}$\,mbar but not stop recoils from entering the gas cell.

A thin window will be installed at the entrance of the gas cell, and supported by a honeycomb structure, to separate these regions. The material and thickness of the window can be varied on a case-to-case basis, but several options have been suggested and simulated, such as 13\,\textmu m-thick Mylar or 4.5\,\textmu m-thick Havar.

\section{Pumping system}
\label{sec:pump}

The planned vacuum system for the facility includes three general types of pumps, three types of gauges and different valves at different points. The whole vacuum diagram can be seen in \autoref{fig:fullsystem}, where the different chambers are shown with their planned pumps, valves and gauges.

\begin{figure}[h]
    \centering
    \includegraphics[width=1\textwidth]{pumpdia2.pdf}
     \caption[Full vacuum system for MARA-LEB]{Full vacuum system diagram for the MARA-LEB facility, including pumps, gates and gauges to be used. The differential pumping region is highlighted in orange. The inset shows a legend for the diagram.}
     \label{fig:fullsystem}
 \end{figure}

The vacuum components for each chamber have been selected specifically for their target pressures, which will not all be the same due to the presence of the differential pumping section, whose chambers can be seen in \autoref{fig:diff_sect}. 
\newpage

The gas cell chamber, which will house the gas cell and the 90º-bent RFQ, is planned to have pressures ranging from 0.1 to 0.01\,mbar, which means that the gas inside will be in a viscous regime. The second chamber, with the straight RFQ, will hold pressures from $10^{-2}$ down to $10^{-6}$\,mbar, having gas in an intermediate to molecular regime within it. The rest of the chambers (including the extraction chamber), will be maintained at a pressure of $10^{-6}$\,mbar, in a molecular regime. 


 \begin{figure}[H]
     \centering
      \includegraphics[width=\textwidth]{LEBdiff.pdf}
      \caption[Differential pumping chamber distribution]{Distribution of the MARA-LEB facility's first components in the different vacuum chambers that constitute the differential pumping section: 1. Gas Cell Chamber, 2. Second Chamber, 3. Extraction Chamber.}
      \label{fig:diff_sect}
  \end{figure}

\subsection{Pumps}
\label{subsec:pumps}
The most important components for the evacuation of the system and the maintainance of the vacuum level are the pumps. The different types and models of pumps chosen for installation in MARA-LEB can be seen in \autoref{tab:Pumps}. These have been chosen primarily for their evacuating power and speed, especially for the differential pumping section, whose pumps will have to deal with higher pressures even after the initial pumpdown. In other sections, where power is less crucial due to more stable conditions, pumps have been selected optimising for cost.

\begin{table}[h!]
    \caption[MARA-LEB Pumps]{Different types of pumps planned for use in the MARA-LEB facility, organised by location and showing their pumping speed for \ce{^{}N_{2}}.}
     \label{tab:Pumps}
    \centering
    \hspace*{-0.5em}
    \begin{tabular}{@{}lllc@{}}
    \hline
    \multicolumn{1}{c}{\textbf{Pump}} & \multicolumn{1}{c}{\textbf{Pump}}  & \multicolumn{1}{c}{\textbf{Pump}} & \multicolumn{1}{c}{\textbf{Pumping}} \\
    \multicolumn{1}{c}{\textbf{Location}} & \multicolumn{1}{c}{\textbf{Type}} & \multicolumn{1}{c}{\textbf{Model}} & {\textbf{Speed [L/s]}} \\
    \hline
    Gas Cell Chamber    & Screw     & GXS750/4200   & 960   \\ 
                                                            \\
    Second Chamber      & Turbo     & STP-iXR2206   & 2200  \\
                        & Scroll    & XDS35i        & 10    \\ 
                                                            \\
    Extraction Chamber  & Turbo     & STP-iXR1606   & 1000  \\
                        & Scroll    & XDS35i        & 10    \\
                                                            \\
    Diagnostics Box 1   & Turbo     & nEXT400D      & 400   \\
                        & Scroll    & nXDS6i        & 1.7   \\
                        & Scroll    & nXDS15i       & 5     \\
                                                            \\
    Diagnostics Box 2   & Turbo     & nEXT400D      & 400   \\
                        & Scroll    & nXDS6i        & 1.7   \\
                                                            \\
    Detector Station    & Turbo     & nEXT400D      & 400   \\
                        & Scroll    & nXDS6i        & 1.7   \\
                        & Scroll    & nXDS15i       & 5     \\
    \hline
    \end{tabular}
 \end{table}

Every vacuum chamber except the gas cell chamber is maintained at vacuum by a turbomolecular (turbo) pump. These pumps are suited for molecular and intermediate regimes, which will be present in these chambers after an initial pumpdown. Every turbo pump is backed by a scroll pump. Due to the way in which turbomolecular pumps work, they need low pressure on the exhaust side, which is provided by the scroll pumps.

The second and extraction chambers will have slightly more powerful turbo and scroll pumps than the chambers further down the beamline, to better deal with their higher planned pressures, resulting from them being in the differential pumping section.

The gas cell chamber, being at a relatively high pressure even after an initial pumpdown, will retain a viscous regime during operation. For this reason, a screw pump was selected. This type of pump is optimal for a viscous regime, unlike the turbo pumps that are used elsewhere in MARA-LEB, which are better suited to intermediate to molecular gas regimes.

Finally, there are two additional scroll pumps, more powerful than their backing counterparts. They are connected directly to the first diagnostics box and the detector station, as seen in \autoref{fig:fullsystem}. These pumps can be used for an initial pumpdown: valves along the beamline are opened (except the one connecting to the differential pumping section) and these scroll pumps take the system from atmospheric pressure to the range of operation of the turbomolecular pumps ($10^{-2} - 10^{-3}\,$mbar). At this point, these scroll pumps are turned off and the turbo pumps are connected, taking this section of the system to $\leq10^{-6}\,$mbar. 

The differential pumping section has to be pumped down independently due to its different conditions. In particular, the gas cell requires baking every time it is replaced, to remove contaminants. This process introduces a constant flow of gas into the gas cell chamber, which would disrupt the pumping of the rest of the system. 


\subsection{Gauges}
\label{subsec:gauges}

Another key component of the vacuum system is its gauge setup. Gauges allow for the monitoring of pressure at different points in the beamline. This is crucial to make decisions on when to change pumps or to detect problems within the facility, such as leaks. 

In MARA-LEB, every turbo pump is backed by a scroll pump, as explained in \autoref{subsec:pumps}. At every connection between a scroll and a turbo pump, there will be a Pirani gauge. These will monitor the vacuum level produced by the scroll pumps to allow for the correct functioning of the turbo pumps. Pirani gauges are suited to the intermediate gas regime produced by the scroll pumps, as they are thermal conductivity gauges, which must operate with a moderate to high gas density. 

Additionally, every zone of the regular pumping section which can be isolated by valves is monitored by a Full Range gauge. Each individual chamber in the differential pumping section will also be fitted with one of these gauges. These gauges are a two-in-one gauge: they combine a Pirani and a cold cathode gauge. This way, higher pressures (viscous to intermediate regime) can be measured by the Pirani gauge and lower pressures (intermediate to molecular regime) can be registered by the cold-cathode gauge.

The gas cell chamber also is planned to have a capacitive gauge with a 1000\,mbar range. This will allow the higher pressures that will be present in the gas cell chamber to be correctly monitored.

\subsection{Valves}
\label{subsec:valves}

Valves are what allows different sections of the system to be isolated from one another, as well as to connect to pumps. They are vital for the control of the vacuum level in the facility.

The valves that connect chambers between them and to pumps will be simple gate valves. These can be either manually or electronically operated. They work by physically blocking the connecting orifice with a barrier. Venting valves will also be installed at different points along the beamline to allow for the re-introduction of air into the vacuum chambers when needed.

The screw pump connected to the gas cell chamber will be regulated by a motorised pendulum valve. This will allow the conductance between the chamber and the pump to be varied so that the effective pumping speed at the chamber changes. This will make the pressure inside this chamber easily regulatable. Pendulum valves work by partially opening or closing an aperture to a desired size, allowing the controller to restrict or allow gasflow as desired.

\end{document}

